\documentclass[a4paper,12pt]{article}


\usepackage[top = .3in,
            right = .2in,
            bottom = .3in,
            left = .2in]{geometry}
\usepackage[UTF8]{ctex}
\usepackage{verbatimbox}
\usepackage{hyperref,xcolor}
\usepackage{amsmath,amssymb}
\usepackage{paralist}
\usepackage{fancyvrb}
\usepackage{minted}
\usepackage{lipsum}



\definecolor{myblue}{rgb}{0.5, 0.0, 1.0}
\hypersetup{colorlinks=true,urlcolor=myblue,}

\newminted{java}{gobble=0,linenos=false,frame=single,breaklines=true,obeytabs=true,tabsize=2,escapeinside=\#\#}
\newminted{html}{gobble=0,linenos=false,frame=single,breaklines=true}
\newminted{sql}{gobble=0,linenos=false,frame=single,breaklines=true,obeytabs=true,tabsize=2,escapeinside=\#\#}

\usemintedstyle[sql]{vs} % autumn

\begin{document}
%\begin{CJK}{UTF8}{gbsn}
\fontsize{14pt}{15.6pt}
\selectfont

\begin{enumerate}

\item sql: Combining AND, OR and NOT

The following SQL statement selects all fields from ``Customers" where country is NOT ``Germany" and NOT ``USA":
\begin{sqlcode}
SELECT * FROM Customers
WHERE NOT Country='Germany' AND NOT Country='USA';
\end{sqlcode}

下面的sql语句不行
\begin{sqlcode}
SELECT * FROM Customers 
WHERE NOT (Country='Germany' AND Country='USA');
\end{sqlcode}

\item Child Classes in Arrays and ArrayLists\\
Noodle.java
\begin{javacode}
class Noodle {  
  protected double lengthInCentimeters;
  protected double widthInCentimeters;
  protected String shape;
  protected String ingredients;
  protected String texture = "brittle";
  
  Noodle(double lenInCent, double wthInCent, String shp, String ingr) {    
    this.lengthInCentimeters = lenInCent;
    this.widthInCentimeters = wthInCent;
    this.shape = shp;
    this.ingredients = ingr;    
  }
  
  public String getCookPrep() {    
    return "Boil noodle for 7 minutes and add sauce.";    
  }
    
  public static void main(String[] args) {    
    Noodle spaghetti, ramen, pho;
    
    spaghetti = new Spaghetti();
    ramen = new Ramen();
    pho = new Pho();
    
    // Add your code below:
    Noodle[] allTheNoodles = {spaghetti, ramen, pho};
    for (Noodle noodle : allTheNoodles) {
      System.out.println(noodle.getCookPrep());
    }
  }  
}
\end{javacode}

Spaghetti.java
\begin{javacode}
class Spaghetti extends Noodle {  
  Spaghetti() {    
  	super(30.0, 0.2, "round", "semolina flour");    
  }
  
  @Override
  public String getCookPrep() {    
    return "Boil spaghetti for 8 - 12 minutes and add sauce, cheese, or oil and garlic.";    
  }  
}
\end{javacode}

Ramen.java
\begin{javacode}
class Ramen extends Noodle {  
  Ramen() {    
    super(30.0, 0.3, "flat", "wheat flour");    
  }
  
  @Override
  public String getCookPrep() {    
    return "Boil ramen for 5 minutes in broth, then add meat, mushrooms, egg, and vegetables.";    
  }  
}
\end{javacode}

Pho.java
\begin{javacode}
class Pho extends Noodle {  
  Pho() {    
    super(30.0, 0.64, "flat", "rice flour");    
  }
  
  @Override
  public String getCookPrep() {    
    return "Soak pho for 1 hour, then boil for 1 minute in broth. Then garnish with cilantro and jalapeno.";    
  }  
}
\end{javacode}
run, the result will be:
\begin{verbatim}
Boil spaghetti for 8 - 12 minutes and add sauce, cheese, or oil and garlic.
Boil ramen for 5 minutes in broth, then add meat, mushrooms, egg, and 
vegetables.
Soak pho for 1 hour, then boil for 1 minute in broth. Then garnish with 
cilantro and jalapeno.
\end{verbatim}

\item Using a Child Class as its Parent Class\\
Dinner.java
\begin{javacode}
class Dinner {  
  private void makeNoodles(Noodle noodle, String sauce) {    
    noodle.cook();
    
    System.out.println("Mixing " + noodle.texture + " noodles made from " + noodle.ingredients + " with " + sauce + ".");
    System.out.println("Dinner is served!");    
  }
  
  public static void main(String[] args) {    
    Dinner noodlesDinner = new Dinner();
    // Add your code here:
    Noodle biangBiang = new BiangBiang();
    noodlesDinner.makeNoodles(biangBiang, "soy sauce and chili oil");    
  }  
}
\end{javacode}

Noodle.java
\begin{javacode}
class Noodle {  
  protected double lengthInCentimeters;
  protected double widthInCentimeters;
  protected String shape;
  protected String ingredients;
  protected String texture = "brittle";
  
  Noodle(double lenInCent, double wthInCent, String shp, String ingr) {    
    this.lengthInCentimeters = lenInCent;
    this.widthInCentimeters = wthInCent;
    this.shape = shp;
    this.ingredients = ingr;    
  }
  
  public void cook() {    
    this.texture = "cooked";    
  }  
}
\end{javacode}

BiangBiang.java
\begin{javacode}
class BiangBiang extends Noodle {  
  BiangBiang() {    
    super(50.0, 5.0, "flat", "high-gluten flour, salt, water");    
  }  
}
\end{javacode}
run, the result will be:
\begin{verbatim}
Mixing cooked noodles made from high-gluten flour, salt, water with soy 
sauce and chili oil.
Dinner is served!
\end{verbatim}

\item Method Overriding\\
Noodle.java
\begin{javacode}
class Noodle {  
  protected double lengthInCentimeters;
  protected double widthInCentimeters;
  protected String shape;
  protected String ingredients;
  protected String texture = "brittle";
  
  Noodle(double lenInCent, double wthInCent, String shp, String ingr) {    
    this.lengthInCentimeters = lenInCent;
    this.widthInCentimeters = wthInCent;
    this.shape = shp;
    this.ingredients = ingr;    
  }
  
  public void cook() {
    System.out.println("Boiling.");
    
    this.texture = "cooked";
  }
  
  public static void main(String[] args) {    
    Spaetzle kaesespaetzle = new Spaetzle();
    kaesespaetzle.cook();
    System.out.println(kaesespaetzle.texture);    
  }  
}
\end{javacode}
Spaetzle.java
\begin{javacode}
class Spaetzle extends Noodle {  
  Spaetzle() {    
    super(3.0, 1.5, "irregular", "eggs, flour, salt");
    this.texture = "lumpy and liquid";    
  }
  
  // Add the new cook() method below:
  @Override
  public void cook() {
    System.out.println("Grinding or scraping dough.");
    System.out.println("Boiling.");
    
    this.texture = "raw";
  }  
}
\end{javacode}
run, the result will be:
\begin{verbatim}
Grinding or scraping dough.
Boiling.
raw
\end{verbatim}

\item Inheriting the Constructor\\
Noodle.java
\begin{javacode}
class Noodle {
  double lengthInCentimeters;
  double widthInCentimeters;
  String shape;
  String ingredients;
  String texture = "brittle";
  
  Noodle(double lenInCent, double wthInCent, String shp, String ingr) {
    this.lengthInCentimeters = lenInCent;
    this.widthInCentimeters = wthInCent;
    this.shape = shp;
    this.ingredients = ingr;
  }
  
  public void cook() {
    this.texture = "cooked";
  }
  
  public static void main(String[] args) {
    Pho phoChay = new Pho();
    System.out.println(phoChay.shape);
  }
}
\end{javacode}

Pho.java
\begin{javacode}
class Pho extends Noodle {
  public Pho() {
    super(30.0, 0.64, "flat", "rice flour");
  }
}
\end{javacode}
run, the result will be: flat

\item Switch Statement
\begin{javacode}
String course = "History";

switch (course) {
  case "Algebra": 
    // Enroll in Algebra
    break; 
  case "Biology": 
    // Enroll in Biology
    break;
  case "History": 
    // Enroll in History
    break;
  case "Theatre":
    // Enroll in Theatre
    break;
  default:
    System.out.println("Course not found");
}
\end{javacode}

\item Learn Java: Methods Review
\begin{javacode}
public class SavingsAccount {

  int balance;

  public SavingsAccount(int initialBalance) {
    balance = initialBalance;
  }

  //Check balance:
  public void checkBalance() {
    System.out.println("Hello!");
    System.out.println("Your balance is " + balance + "\n");
  }

  //Deposit:
  public void deposit(int amountToDeposit) {
    int afterDeposit = balance + amountToDeposit;
    balance = afterDeposit;
    System.out.println("You just deposited " + amountToDeposit);
  }

  //Withdrawing:
  public int withdraw(int amountToWithdraw) {
    int afterWithdraw = balance - amountToWithdraw;
    balance = afterWithdraw;
    System.out.println("You just withdrew " + amountToWithdraw);
    return amountToWithdraw;
  }

  public static void main(String[] args) {
    SavingsAccount savings = new SavingsAccount(2000);

    //Check balance:
    savings.checkBalance();

    //Withdrawing:
    savings.withdraw(300);
    //Check balance:
    savings.checkBalance();

    //Deposit:
    savings.deposit(600);
    //Check balance:
    savings.checkBalance();

    //Deposit:
    savings.deposit(600);
    //Check balance:
    savings.checkBalance();
  }
}
\end{javacode}

\item The \verb|toString()| Method
\begin{javacode}
public class Store {
  // instance fields
  String productType;
  double price;
  
  // constructor method
  public Store(String product, double initialPrice) {
    productType = product;
    price = initialPrice;
  }

  // main method
  public static void main(String[] args) {
    Store lemonadeStand = new Store("Lemonade", 3.75);
    Store cookieShop = new Store("Cookies", 5);
    System.out.println(lemonadeStand);
    System.out.println(cookieShop);
  }
  	
  public String toString() {
    return "This store sells " + productType + " at a price of " + price + ".";
  }
}
\end{javacode}
solves \verb|System.out.println(lemonadeStand);| only prints something like \verb|Store@6bc7c054|, where Store is the name of the class and 6bc7c054 is its position in memory.

\item Classes: Assigning Values to Instance Fields
\begin{javacode}
public class Car {
  String color;

  public Car(String carColor) {
    // assign parameter value to instance field
    color = carColor;
  }

  public static void main(String[] args) {
    // parameter value supplied when calling constructor
    Car ferrari = new Car("red");
		// accessing a field: objectName.fieldName
    System.out.println(ferrari.color);
  }
}
\end{javacode}
The type of the value passed to the constructor \textbf{must match} the type declared by the parameter.

\item Classes: Constructor Parameters
\begin{javacode}
public class Car {
  String color;

  // constructor method with a parameter
  public Car(String carColor) {
    // parameter value assigned to the field
    color = carColor;
  }
  public static void main(String[] args) {
    // program tasks
  }
}
\end{javacode}

\item Classes: Instance Fields
\begin{javacode}
public class Car {
  /*
  declare fields inside the class
  by specifying the type and name
  */
  String color;

  public Car() {
    /* 
    instance fields available in
    scope of constructor method
    */
  }

  public static void main(String[] args) {
    // body of main method
  }
}
\end{javacode}

\item Classes: Constructors
\begin{javacode}
public class Store {
  
  // new method: constructor!
  public Store() {
    System.out.println("I am inside the constructor method.");
  }
  
  // main method is where we create instances!
  public static void main(String[] args) {
    System.out.println("Start of the main method.");
    
    // create the instance below
    Store lemonadeStand = new Store();
    // print the instance below
    System.out.println(lemonadeStand);
  }
}
\end{javacode}
The constructor, \verb|Store()|, shares a name with the class. 

Variables that reference an instance have a type of the class name. We create instances by \textit{calling or invoking} the constructor within \verb|main()|.

Review the order of the printed messages:
\renewcommand{\labelenumii}{$\diamond$}
\begin{enumerate}
\item Running the program invokes \verb|main()|

\item We create an instance so we move from \verb|main()| to \verb|Store()|

\item The code inside \verb|Store()| runs

\item When \verb|Store()| finishes execution, we return to \verb|main()|

\end{enumerate}


\end{enumerate}



%\end{CJK}
\end{document}