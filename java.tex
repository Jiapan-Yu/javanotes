\documentclass[a4paper,12pt]{article}


\usepackage[top = .3in,
            right = .2in,
            bottom = .3in,
            left = .2in]{geometry}
\usepackage[UTF8]{ctex}
\usepackage{verbatimbox}
\usepackage{hyperref,xcolor}
\usepackage{amsmath,amssymb}
\usepackage{paralist}
\usepackage{fancyvrb}
\usepackage{minted}
\usepackage{lipsum}



\definecolor{myblue}{rgb}{0.5, 0.0, 1.0}
\hypersetup{colorlinks=true,urlcolor=myblue,}

\newminted{java}{gobble=0,linenos=false,frame=single,breaklines=true,obeytabs=true,tabsize=2,escapeinside=\#\#}
\newminted{html}{gobble=0,linenos=false,frame=single,breaklines=true}

\begin{document}
%\begin{CJK}{UTF8}{gbsn}
\fontsize{14pt}{15.6pt}
\selectfont

\begin{enumerate}
\item Classes: Constructors
\begin{javacode}
public class Store {
  
  // new method: constructor!
  public Store() {
    System.out.println("I am inside the constructor method.");
  }
  
  // main method is where we create instances!
  public static void main(String[] args) {
    System.out.println("Start of the main method.");
    
    // create the instance below
    Store lemonadeStand = new Store();
    // print the instance below
    System.out.println(lemonadeStand);
  }
}
\end{javacode}
The constructor, \verb|Store()|, shares a name with the class. We create instances by \textit{calling or invoking} the constructor within \verb|main()|.

Review the order of the printed messages:
\renewcommand{\labelenumii}{$\diamond$}
\begin{enumerate}
\item Running the program invokes \verb|main()|

\item We create an instance so we move from \verb|main()| to \verb|Store()|

\item The code inside \verb|Store()| runs

\item When \verb|Store()| finishes execution, we return to \verb|main()|

\end{enumerate}


\end{enumerate}



%\end{CJK}
\end{document}