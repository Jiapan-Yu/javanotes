\documentclass[a4paper, 12pt]{article}

\usepackage[UTF8]{ctex}
\usepackage{graphicx}
\usepackage{lipsum}
\usepackage[colorlinks=true]{hyperref}
\usepackage[margin=.5in]{geometry}

\begin{document}
\pagenumbering{gobble}

\large
\title{Java}
\author{Joshua Yu}
\date{11 July. 2019}
\maketitle
\tableofcontents

%\lipsum[1-5]

\section{Miscellaneous}

\begin{enumerate}
\item MySql用得比较多

\item 在谈到Java开发人员经常使用的Java版本时,Java 8是胜利者。但应该记住,在2019年1月之后,Java 8将不再有公开更新。谈到Java开发人员最经常使用的IDE/编辑器时,IntelliJ IDEA看起来是明显的赢家。 \url{https://www.sohu.com/a/235541971_100159565}

\item winscp下载地址:\url{https://winscp.net/eng/download.php}

\item putty下载地址:\url{https://www.chiark.greenend.org.uk/~sgtatham/putty/latest.html}

\end{enumerate}

\section{Hello World}
\subsection{Hello Java File!}
HelloYou.java

\begin{verbatim}
public class HelloYou {
  public static void main(String[] args) {
    System.out.println("Hello Joshua!");
  }
}
\end{verbatim}
\verb|println| is short for ``print line''. It has to be double quotes inside \verb|println|

\subsection{Commenting Code}
When comments are short we use the single-line syntax: \verb|//|.

When comments are long we use the multi-line syntax: \verb|/*| and \verb|*/|.

\subsection{Semicolons and Whitespace}
Java does not interpret \textit{whitespace}, the areas of the code without syntax, but humans use whitespace to read code without difficulty.

Java \textbf{does} interpret semicolons. Semicolons are used to mark the end of a statement, one line of code that performs a single task.

Let's contrast statements with the curly brace, \verb|{}|. Curly braces mark the scope of our classes and methods. There are no semicolons at the end of a curly brace.

\subsection{Compilation: Catching Errors}
with a file called Plankton.java, we could compile it with the terminal command:

\verb|javac Plankton.java|

A successful compilation produces a \verb|.class| file: \verb|Plankton.class|, that we execute with the terminal command:

\verb|java Plankton|




\end{document}




