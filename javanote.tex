\documentclass[a4paper, 12pt]{article}

\usepackage[UTF8]{ctex}
\usepackage{graphicx}
\usepackage{lipsum}
\usepackage[colorlinks=true]{hyperref}
\usepackage[margin=.5in]{geometry}

\begin{document}
\pagenumbering{gobble}

\large
\title{Java}
\author{Joshua Yu}
\date{11 July. 2019}
\maketitle
\tableofcontents

%\lipsum[1-5]

\section{Caveats}
\begin{itemize}
\item ``Intellij IDEA run button grayed out''?
\begin{enumerate}
\item Right click the folder containing your source
\item Select Mark Directory as $\rightarrow$ Test Source Root
\end{enumerate}


\end{itemize}


\section{Hello World}
\subsection{Hello Java File!}
HelloYou.java

\begin{verbatim}
public class HelloYou {
  public static void main(String[] args) {
    System.out.println("Hello Joshua!");
  }
}
\end{verbatim}
\verb|println| is short for ``print line''. It has to be double quotes inside \verb|println|

\subsection{Commenting Code}
When comments are short we use the single-line syntax: \verb|//|.

When comments are long we use the multi-line syntax: \verb|/*| and \verb|*/|.

\subsection{Semicolons and Whitespace}
Java does not interpret \textit{whitespace}, the areas of the code without syntax, but humans use whitespace to read code without difficulty.

Java \textbf{does} interpret semicolons. Semicolons are used to mark the end of a statement, one line of code that performs a single task.

Let's contrast statements with the curly brace, \verb|{}|. Curly braces mark the scope of our classes and methods. There are no semicolons at the end of a curly brace.

\subsection{Compilation: Catching Errors}
with a file called Plankton.java, we could compile it with the terminal command:

\verb|javac Plankton.java|

A successful compilation produces a \verb|.class| file: \verb|Plankton.class|, that we execute with the terminal command:

\verb|java Plankton|


\section{Java Program Structure}
Java is a case-sensitive language.

\subsection{Classes}
All Java programs are made of at least one class. The class name must match the file: our file is \textbf{HelloWorld.java} and our class is \verb|HelloWorld|. We capitalize every word, a style known as \textit{pascal case}. Java variables and methods are named in a similar style called \textit{camel case} where every word \textbf{after the first} is capitalized.

\subsection{Methods}
Every Java program must have a method called \verb|main()|. A method is a sequence of tasks for the computer to execute. This \verb|main()| method holds all of the instructions for our program.

\href{https://github.com/twitter/commons/blob/master/src/java/com/twitter/common/styleguide.md#dont-embed-metadata-in-variable-names}{Don't embed metadata in variable names} seems differenct from JavaScript


\section{Variables}
\textit{Primitive datatypes} are types of data built-in to the Java system

\verb|int double boolean char|

The \verb|char| data type which store \textit{single character} can hold any character, like a letter, space, or punctuation mark. It must be surrounded by single quotes, '. For example:
\verb|char punctuation = '!';|

So far, we have learned primitive data types, which are the simplest types of data with no built-in behavior. Our programs will also use \verb|String|s, which store multiple characters using double quotes, which are objects, instead of primitives. Objects have built-in behavior.

A variable starts with a valid letter, or a \verb|$|, or a \verb|_|.


\section{Manipulating Variables}
With objects, such as \verb|String|s, we can't use the primitive equality operator. To test equality with \verb|String|s, we use a built-in method called \verb|.equals()|.
\subsection{String Concatenation}
The \verb|+| operator can be used to concatenate \verb|String|s.
\begin{verbatim}
// can be primitive datatype as well, Java will intelligently make it a 
// String first
String username = "PrinceNelson";
System.out.println("Your username is: " + username);
\end{verbatim}


\section{Introduction to Classes}
\subsection{Classes: Syntax}
The fundamental concept of object-oriented programming is the class. Every program has at least one class
\begin{verbatim}
public class Car {
// scope of Car class starts after curly brace

  public static void main(String[] args) {
    // scope of main() starts after curly brace

    // program tasks

  }
  // scope of main() ends after curly brace

}
// scope of Car class ends after curly brace
\end{verbatim}
This example defines a \verb|class| named \verb|Car|. \verb|public| is an \textit{access level modifier} that allows \textbf{other} classes to interact with this class.

\subsection{Classes: Constructors}
We create objects (instances of a class) using a constructor method. See java.pdf for an example

\subsection{Classes: Instance Fields}
We'll add associated data to an object by introducing instance variables, or instance fields. Instance fields are the state in our objects. See java.pdf for an example

Instance Fields are also available inside all methods in the class, since it is in the scope of the whole class.

\subsection{Classes: Constructor Parameters}
Parameters specify the type and name of data available for reference within a method's scope. See java.pdf for an example

\subsection{Classes: Assigning Values to Instance Fields}
Now that our constructor has a parameter, we must pass values into the method call. These values become the state of the instance. See java.pdf for an example


\section{Methods}
\subsection{Defining Methods}
\begin{verbatim}
public void startEngine() {
  System.out.println("Starting the car!");
  System.out.println("Vroom!");
}
\end{verbatim}
The first line, \verb|public void startEngine()|, is the method signature. It gives the program some information about the method:
\begin{itemize}
\item \verb|public| means that other classes can access this method

\item The \verb|void| keyword means that there is no specific output from the method (no value is returned)

\item \verb|startEngine()| is the name of the method

\end{itemize}

\subsection{Calling Methods}
When we add the \verb|startEngine()| method to the \verb|Car| class, it becomes available to use on any \verb|Car| object

\subsection{Reassigning Instance Fields}
In this codecademy example, \verb|price| is an instance field of type \verb|double|
\begin{verbatim}
// increase price method
public void increasePrice(double priceToAdd){
  double newPrice = price + priceToAdd;
  price = newPrice;
}
\end{verbatim}
why not just \verb|price = price + priceToAdd;|? 改成这个编译的时候也没有报错

\subsection{Returns}
We can use a value outside of the method it was created in if we \textit{return} it from the method.
\begin{verbatim}
public int numberOfTires() {
   int tires = 4;
   return tires;
}
\end{verbatim}
In past exercises, when creating new methods, we used the keyword \verb|void|. Here, we are replacing \verb|void| with \verb|int|, to signify that the \textit{return type} is an \verb|int|.

\subsection{The toString() Method}
When we print out Objects, we often see a String that is not very helpful in determining what the Object represents: \verb|Store@6bc7c054|.

See java.pdf for an example


\section{Introduction to Control Flow}
\textit{if-then-else} statement:
\begin{itemize}
\item If \textit{condition} is true, then do something

\item Else, do a different thing

\end{itemize}

Only one of the code blocks will run.

\vspace{1.3em}

Switch Statement:

See java.pdf for an example. Switch blocks are different than other code blocks because they are not marked by curly braces and we use the \verb|break| keyword to exit the switch statement.

\subsection{Conditional Operators}
Conditional-AND, \verb|&&|, evaluates to \verb|true| if the booleans on both sides are \verb|true|.

Conditional-OR, $\vert\vert$, evaluates to \verb|true| if one or both of the booleans on either side is \verb|true|.

Logical-NOT, \verb|!|, evaluates to the opposite boolean value to which it is applied.


\section{Miscellaneous}

\begin{enumerate}
\item MySql用得比较多

\item 在谈到Java开发人员经常使用的Java版本时,Java 8是胜利者。但应该记住,在2019年1月之后,Java 8将不再有公开更新。谈到Java开发人员最经常使用的IDE/编辑器时,IntelliJ IDEA看起来是明显的赢家。 \url{https://www.sohu.com/a/235541971_100159565}

\item winscp下载地址:\url{https://winscp.net/eng/download.php}

\item putty下载地址:\url{https://www.chiark.greenend.org.uk/~sgtatham/putty/latest.html}

\end{enumerate}

\end{document}




